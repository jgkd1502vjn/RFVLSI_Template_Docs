\documentclass[rfvlsi_template_jrnl.tex]{subfiles}
\begin{document}

\section{Introduction}
% The very first letter is a 2 line initial drop letter followed
% by the rest of the first word in caps.
% 
% form to use if the first word consists of a single letter:
% \IEEEPARstart{A}{demo} file is ....
% 
% form to use if you need the single drop letter followed by
% normal text (unknown if ever used by IEEE):
% \IEEEPARstart{A}{}demo file is ....
% 
% Some journals put the first two words in caps:
% \IEEEPARstart{T}{his demo} file is ....
% 
% Here we have the typical use of a "T" for an initial drop letter
% and "HIS" in caps to complete the first word.

%Based on the discussions in Section II, we categorize those architectures according to their idealistic design concepts. 

\IEEEPARstart{T}{his} document serves as a starting template for writing IEEE-trans. paper in NCTU, RFVLIS-Lab.

\subsection{First Time Usa of Latex}

\begin{enumerate}
  \item Download Miktex.
  \item \textit{Use} \emph{package} \textsc{manager} \uppercase{to} install the following packages:
	\begin{itemize}
		\item All IEEE transactions/bibtex packages
		\item \textbf{Textcomp}: support some symbols.
		\item \textbf{amsmath}: support some maths. 
		\item \textbf{Subfiles}: support independent compilable subfiles .tex structure as used in this template. 
		\item \textbf{Dblfloatfix}: fixes double column figures ordering problems.
	\end{itemize}
\end{enumerate}

\subsection{Specialized Semiconductor Process}

\begin{itemize}
\item[1] Some forwords about why needs a rectifier?.
\item[2] Focus on introduction of several representative topologies.
\item[3] Bring up the the later sections.
\end{itemize}

\subsection{Equation Templates }

In all of the following approaches, a gate DC bias $V_{g,bias}$ is introduced as in Eq. XXXX.

\begin{equation}
\label{NStageRectVout}
V_{out}=N⋅(V_{RF,Peak}-V_{th}+V_{gs,bias}).
\end{equation}

\subsection{Table Templates}

%%%%%%%%%%%%%%%%%%%%%%%%%%%%%%%%%%%%%%%%%%%%%%%%%%
% Summary table for UHF RF-to-DC rectifiers.
\begin{table*}[!t]
\centering
\caption{Summary of the UHF RF-to-DC rectifier performance.}
\label{table_UHF_performance}
\centering
\begin{IEEEeqnarraybox}[\IEEEeqnarraystrutmode\IEEEeqnarraystrutsizeadd{2pt}{0pt}][b]{v/t/V/t/v/t/v/t/v/t/v}
\IEEEeqnarrayrulerow\\
&\textbf{Specification}&&\textbf{This work}&& \textbf{A}&&\textbf{B}&&\textbf{C}&\\
\IEEEeqnarraydblrulerow\\
&{Frequency(MHz)}&&900&& 950&&915&&915&\\
\IEEEeqnarrayrulerow\\

&{Technology}&&\parbox{20ex}{\raggedright 0.28$\mu$m thick-gate oxide CMOS in 65nm process}&& 0.35$\mu$m&&90nm&&0.2$\mu$m&\IEEEeqnarraystrutsizeadd{8pt}{8pt}\\
\IEEEeqnarrayrulerow\\
&{\parbox{19ex}{PCE@Output power}}&&27.97\%@19.3mW&&15.1\%@0.6$\mu$W&&11\%@13.1$\mu$W&&71.5\%@0.285mW&\IEEEeqnarraystrutsizeadd{4pt}{4pt}\\
\IEEEeqnarrayrulerow\\
&{\parbox{19ex}{Number of stage\\/ type of the rectifier}}&&5/half-wave&&\parbox{15ex}{1 (six stacks)\\/ full-wave}&&17/half-wave&&1/full-wave&\IEEEeqnarraystrutsizeadd{4pt}{4pt}\\
\IEEEeqnarrayrulerow\\
&{Chip area}	&&0.442mm$^2$&&0.104mm$^2$&&0.19mm$^2$&&0.133mm$^2$&\\
\IEEEeqnarrayrulerow%
\end{IEEEeqnarraybox}
\end{table*}

%Note:
% \IEEEeqnarrayrulerow, \IEEEeqnarrayrulerow[rule_thickness] 
% \IEEEeqnarraydblrulerow , \IEEEeqnarraydblrulerow[rule_thickness][spacing]
% \IEEEeqnarraydblrulerowcut, \IEEEeqnarraydblrulerowcut[rule_thickness][spacing]
% Adding spaces above/below each row:  \IEEEeqnarraystrutsizeadd{4pt}{4pt}
% \raggedright,\raggedleft
%\IEEEpubidadjcol

\subsection{Inductive Peaking Approach}



\subsection{Other Approaches}

% You must have at least 2 lines in the paragraph with the drop letter
% (should never be an issue)


\end{document}

