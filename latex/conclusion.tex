\documentclass[rfvlsi_template_jrnl.tex]{subfiles}
\begin{document}

\section{Conclusion}
% The very first letter is a 2 line initial drop letter followed
% by the rest of the first word in caps.
% 
% form to use if the first word consists of a single letter:
% \IEEEPARstart{A}{demo} file is ....
% 
% form to use if you need the single drop letter followed by
% normal text (unknown if ever used by IEEE):
% \IEEEPARstart{A}{}demo file is ....
% 
% Some journals put the first two words in caps:
% \IEEEPARstart{T}{his demo} file is ....
% 
% Here we have the typical use of a "T" for an initial drop letter
% and "HIS" in caps to complete the first word.


\appendices
\section{Proof of Eq.xxx}
\label{sec:mismatch}

\renewcommand{\theequation}{\thesection.\arabic{equation}}
\setcounter{equation}{0}

If there is a mismatch between the two reference MOSTFETs, the $\Delta G_m$ would be derived in this section. Based on the error models shown in Section \ref{subsec:ErrorModels}.  As shown in Fig.\ref{fig:error_curve}, the operating points will be locked by the feedback loop, while the difference of output currents is \textbf{XXXX} and the difference of input voltages is \textbf{xxxx}.

The current equations of different operating points are \textbf{don\rq{t} assume square-law!!!}
\begin{IEEEeqnarray}{R}
I_2=k^x/2 (W/L) (V_2^x-V_t )^2   
\label{eqn:IGRVgEquation}
\end{IEEEeqnarray}
As a result,

\begin{IEEEeqnarray}{R}
XXXXXXXXXXXXXXXXXXXXXXXX
\label{eqn:IGRVgEquation}
\end{IEEEeqnarray}

Define the transconductance which is referred to $M_2$ as 
\begin{IEEEeqnarray}{R}
XXXXXXXXXXX
\label{eqn:IGRVgEquation}
\end{IEEEeqnarray}

Eq.XXX can be expressed as 
%\begin{equation}
%(∆I)_{d_mis}\cdot (+∆I)_d \cdot (+∆I)_{d1}=G_m2^x ((V_1^X-V_2^X )+∆W/2W\cdot (1+(V_1^x-V_2^x)/((V_2^x-V_t ) ))^2 ). 
%\end{equation}
Assuming XXXXX, we have
\begin{IEEEeqnarray}{R}
yyyy.                                            
\label{eqn:IGRVgEquation}
\end{IEEEeqnarray}

Since
\begin{equation}
\end{equation}
$G_{m2}^x$ is obtained as
\begin{equation}
zzzzzzz.                    
\end{equation}

\section{Relationships between small-signal $g_m$ and $1/R$}
\renewcommand{\theequation}{\thesection.\arabic{equation}}
\setcounter{equation}{0}

Taylor\rq{s} expansion of $I_{out}$ at $V_0$ is:
\begin{IEEEeqnarray}{R}
I_{out}(V_0+v)=I_0+\sum_{i=1}^\infty  a_i\cdot v^{i}
\label{eqn:Iout_expansion}
\end{IEEEeqnarray}
, with $a_i$ being the $i$-th coefficient of the expansion.  
 Subtracting $I_{out}(V_0)$ from $I_{out}(V_0+\Delta V)$, and dividing it by $\Delta V$ will result in: 
\begin{IEEEeqnarray}{R}
\frac{\Delta I}{\Delta V} =\sum_{i=1}^\infty  a_i\cdot \Delta V^{i-1}\equiv \frac{1}{R}
\label{eqn:Gm_expansion}
\end{IEEEeqnarray}
\end{document}