\documentclass[bare_jrnl.tex]{subfiles}
\begin{document}

\section{Conclusion}
% The very first letter is a 2 line initial drop letter followed
% by the rest of the first word in caps.
% 
% form to use if the first word consists of a single letter:
% \IEEEPARstart{A}{demo} file is ....
% 
% form to use if you need the single drop letter followed by
% normal text (unknown if ever used by IEEE):
% \IEEEPARstart{A}{}demo file is ....
% 
% Some journals put the first two words in caps:
% \IEEEPARstart{T}{his demo} file is ....
% 
% Here we have the typical use of a "T" for an initial drop letter
% and "HIS" in caps to complete the first word.


The use of IPVM can produce a larger $v_{gs}$ than $v_{ds}$ in a passive manner at resonant frequencies for rectifying transistors.  When IPVM is used to form the IGR, the rectifier will achieve a lower forward resistance, lower reverse leakage current, and lower effective threshold. Each of these properties will improve both the sensitivity and PCE of the rectifier. This is experimentally proved in a 53GHz mmWave rectifier IC which achieves 20\% at 7dBm. IGR is an effective approach to improve sensitivity and PCE in high-frequency RF-to-DC rectifier. The IGR can be implemented in a CMOS process without additional photo-mask, this allows integration of IGR into a complete wireless-powered system with high sensitive and PCE in the future.

\end{document}